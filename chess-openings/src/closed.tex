\begin{minipage}[t]{.175\linewidth}
\fenboard{rnbqkb1r/1p3ppp/p3pn2/2p5/2BP4/4PN2/PP3PPP/RNBQ1RK1 w kq - 0 7}
\raggedright
\begin{center}
\scalebox{.560}{\showboard}
\end{center}
\newgame
%FEN@rnbqkb1r/1p3ppp/p3pn2/2p5/2BP4/4PN2/PP3PPP/RNBQ1RK1 w kq - 0 7
\openingname{Queen's Gambit Accepted}: \mainline{1. d4 d5 2. c4 dxc4}, main line: \mainline{3. Nf3 Nf6 4. e3 e6 5. Bxc4 c5 6. O-O a6} (diagram).\vspace{2mm}
\end{minipage}
\hspace{5mm}
\begin{minipage}[t]{.175\linewidth}
\fenboard{rnbqk2r/ppp1bppp/4pn2/3p2B1/2PP4/2N2N2/PP2PPPP/R2QKB1R b KQkq - 5 5}
\raggedright
\begin{center}
\scalebox{.560}{\showboard}
\end{center}
\newgame
%FEN@rnbqk2r/ppp1bppp/4pn2/3p2B1/2PP4/2N2N2/PP2PPPP/R2QKB1R b KQkq - 5 5
\openingname{Queen's Gambit Declined}: \mainline{1. d4 d5 2. c4 e6}, main line: \mainline{3. Nc3 Nf6 4. Bg5 Be7 5. Nf3} (diagram).\vspace{2mm}
\end{minipage}
\hspace{5mm}
\begin{minipage}[t]{.175\linewidth}
\fenboard{rnbqkb1r/pp2pppp/2p2n2/3p4/2PP4/5N2/PP2PPPP/RNBQKB1R w KQkq - 2 4}
\raggedright
\begin{center}
\scalebox{.560}{\showboard}
\end{center}
\newgame
%FEN@rnbqkb1r/pp2pppp/2p2n2/3p4/2PP4/5N2/PP2PPPP/RNBQKB1R w KQkq - 2 4
\openingname{Slav Defence}: \mainline{1. d4 d5 2. c4 c6}, main line: \mainline{3. Nf3 Nf6} (diagram) followed by \mainline{4. e3} or \variation{4. Nc3}.\vspace{2mm}
\end{minipage}
\hspace{5mm}
\begin{minipage}[t]{.175\linewidth}
\fenboard{r1bqkb1r/pp2pppp/2n2n2/2pp4/3P1P2/2PBP3/PP4PP/RNBQK1NR b KQkq f3 0 5}
\raggedright
\begin{center}
\scalebox{.560}{\showboard}
\end{center}
\newgame
%FEN@r1bqkb1r/pp2pppp/2n2n2/2pp4/3P1P2/2PBP3/PP4PP/RNBQK1NR b KQkq f3 0 5
\openingname{Stonewall Attack}: \mainline{1. d4 d5 2. e3 Nf6 3. Bd3 c5 4. c3 Nc6 5. f4} (regardless of how Black defends!).\vspace{2mm}
\end{minipage}
\hspace{5mm}
\begin{minipage}[t]{.175\linewidth}
\fenboard{r1bq1rk1/pp3ppp/2nbpn2/2pp4/3P4/2PBPN2/PP1N1PPP/R1BQR1K1 b - - 2 8}
\raggedright
\begin{center}
\scalebox{.560}{\showboard}
\end{center}
\newgame
%FEN@r1bq1rk1/pp3ppp/2nbpn2/2pp4/3P4/2PBPN2/PP1N1PPP/R1BQR1K1 b - - 2 8
\openingname{Colle System}: \mainline{1. d4 Nf6 2. Nf3 d5 3. e3 e6 4. Bd3 c5 5. O-O Nc6 6. Re1 Bd6 7. c3 O-O 8. Nbd2} (regardless of how Black defends!).\vspace{2mm}
\end{minipage}
\newline
\begin{minipage}[t]{.175\linewidth}
\fenboard{rnbqkb1r/ppp1pppp/5n2/3p4/3P1B2/5N2/PPP1PPPP/RN1QKB1R b KQkq - 3 3}
\raggedright
\begin{center}
\scalebox{.560}{\showboard}
\end{center}
\newgame
%FEN@rnbqkb1r/ppp1pppp/5n2/3p4/3P1B2/5N2/PPP1PPPP/RN1QKB1R b KQkq - 3 3
\openingname{London System}: \mainline{1. d4 d5 2.Nf3 Nf6 3.Bf4} (diagram, regardless of how Black defends!).
Black usually plays \mainline{3...c5}, \variation{3...e6}, \variation{3...c6} or \variation{3...Bf5} to which \mainline{4. e3} is universal reply.\vspace{2mm}
\end{minipage}
\hspace{5mm}

